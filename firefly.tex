\documentclass[a4paper,12pt]{article}
\begin{document}
\title{Firefly UI Software Specification}
\author{Lukas Gemar, Paul Wells, Graham Dempsey, Kit Werley, Adam Cohen\\
		Q-state Biosciences}
\date{September 10, 2014}
\maketitle

\begin{abstract}
This document's purpose is to serve as a meeting place for discussion of
the second version of the Firefly's user interface (UI). The specification here
should separate itself from the hardware implementation of the microscope
and tie itself more closely to the user experience. The user thinks in terms
stimulus patterns, cells, and their electrical waveforms, so the initial 
discussion will involve an abstract specification of the experimental configuration 
that allows the user to experience the Firefly as a biological oscilloscope 
and function generator. The second part of the document will include a more 
in depth exploration of the parameter space and investigate how to 
abstractly and concretely represent the configuration of a particular 
experiment. The conclusions drawn here will shape both the user 
experience and how the front-end interface interacts with the underlying 
control system.
\end{abstract}
\section{Paradigm: Firefly as a Biological Function Generator and Oscilloscope}
\subsection{Region Selection}
Each region will be contained within the current field of view. 
There will be three ways to define regions:
\begin{itemize}
 \item Choose a preset region.
 \item Define a custom region by clicking on the current field of view.
 \item Use a segmentation algorithm to generate regions.
\end{itemize}
\subsubsection{Preset Regions and Patterns}
Preset regions will include the full field of view and user generated masks.
\subsubsection{User-Defined and Algorithmically-Segmented Regions}
The user will either click on the field of view to select a region or run a 
segmentation algorithm. The user will be able to perform the following 
operations on a region once it has been placed in a field of view:
\begin{itemize}
  \item Resize
  \item Reshape
  \item Drag around the field of view
  \item Delete
\end{itemize}
\subsubsection{Regions as Sources and Probes}
Each region abstractly represents a part of a circuit accessible to a signal
generator or oscilloscope probe. A user designates a region as a Stimulus Region,
Record Region, or both.
\subsection{Stimulus Pattern Generation}
For each region designated as a Stimulus Region, the user can specify a pattern
of stimulation. The user specifies stimulus patterns by selecting among
predefined stimulus waveforms.
\subsection{Electrical Waveform Monitor}
For each region designated as a Record Region, the user can specify an algorithm
for live analysis and monitoring. The user specifies the monitor algorithm by 
selecting among predefined options.
\section{Implementation: Firefly UI Parameter Space and Experimental Configuration}
\subsection{Sample Selection}
A single plate may hold one or more samples. An index $i$ specifies which 
sample is currently in view and $n$ is the number of samples.
\subsubsection{Plate Geometry}
The user will select the plate geometry, specifying the number of samples $n$.
\subsubsection{Automation}
The user or an automated program can select sample $i$ for viewing.
\subsection{Field of View Selection}
Once a sample is selected, the position of the field of view within the sample
is specified by an $(x,y,z)$ coordinate.
\subsection{Spatial Pattern Generation}
These spatial patterns are analogous to the "regions" described in the Paradigm
section above.
\subsubsection{Statically Built Patterns}
These are two dimensional masks in the coordinate space of the camera. The 
simplest such pattern will be the full field of view of the camera. The user
will have the option of uploading the masks in camera or DMD coordinates, but 
the internal representation will be in camera coordinates and the user will
visualize these spatial patterns overlaid on the current field of view.
\subsubsection{Dynamically Built Patterns}
These are patterns specified by clicking around a region or a segmentation 
algorithm at run-time. Each of these dynamically selected patterns can be 
manipulated in the following ways:
\begin{itemize}
  \item Resize
  \item Reshape
  \item Drag around the field of view
  \item Delete
\end{itemize}
\subsection{Temporal Pattern Generation}
\subsubsection{Static Patterns}
Static patterns are created by a user before run-time of the program.
\paragraph{Pattern Upload}
A user will upload a pattern in a standard format such as a text file.
The user will also give the pattern a name at this time for easy 
reference later.
\paragraph{Pattern Visualization}
After a user uploads the pattern, there will be an interface for viewing
the uploaded waveform to determine that the waveform is correct.
\subsubsection{Dynamic Patterns}
A user will create some temporal patterns in real time. These patterns
are generated by user input: toggling a switch on and off to create a 
digital waveform in real time or entering a float in a text box to change
an analog value.
\subsection{Device / Monitor Panel}
There will be a panel onto which a user can drag and drop Devices and Monitors. 
The panel will then show a configuration window for each Device and Monitor.
\subsubsection{Devices}
Each device will have a name, set of parameters, and input channels. This
will allow for the addition of new wavelengths, filters, etc.
\subsubsection{Monitors}
Each monitor is specified by the algorithm it applies the stream
of data that arrives on its input channel. Monitors will implement analysis
algorithms for generating real-time data visualizations.
\subsection{I/O Configuration}
\subsubsection{Mapping Output Channels to Devices}
The user will create a mapping between control channels to Device
input channels.
\subsubsection{Mapping Input Channels to Monitors}
The user will create a mapping between sampling channels and Monitor
input channels.
\subsubsection{Mapping Input and Output Channels to Temporal Patterns}
The user will have the freedom to create arbitrary mappings between
control and sampling channels and temporal waveforms. This will accommodate
temporal control of new pieces of hardware and sampling of new kinds
of data from detectors.
\end{document}
